\documentclass{article}
\pagenumbering{gobble}
\usepackage[margin=.35in]{geometry}
\usepackage{enumitem}
\usepackage{longtable}

\begin{document}
{\noindent \Large \textsc{FCMkII DESIGN SCRATCH}}
\vspace{1em}

{\par\noindent\large\textsc{Communications}}

\small

\renewcommand{\arraystretch}{1.5}
{
\begin{center}
\begin{longtable}{|p{.92\textwidth}|}
	\hline
	\textsc{Standardized Port Numbers}\\
	\hline
	\begin{description}
		\item[S. Listener (Broadcast):]\texttt{65000}
		\item[S. MISO: ]\texttt{65001}
		\item[S. MOSI: ]\texttt{65002}
	\end{description}\\
	\hline
\end{longtable}
\end{center}
}

\begin{center}
\small
\begin{longtable}{| p{.45\textwidth} | p{.45\textwidth} |}
	\hline
	\multicolumn{2}{|c|}{\textsc{Breakdown}}\\
	\hline
	\textbf{Master} & \textbf{Slave}\\
	\hline
	\multicolumn{2}{|c|}{\textbf{Parts}}\\
	\hline
	\begin{itemize}[leftmargin=*]
		\item Broadcast Thread
			\begin{itemize}
				\item Standard broadcast
				\item Update broadcast
				\item Shutdown broadcast
			\end{itemize}
		\item {Listener\hspace{.5em}thread\hspace{.5em}}
			\linebreak NOTE: Here, responding Slaves can categorized are known and unknown
			\begin{itemize}
				\item List new, unknown Slaves for adding
				\item Mark known, disconnected Slaves for reconnection
				\item Ignore messages from connected Slaves
				\item Send startup messages to Bootloaders when appropriate
			\end{itemize}
		\item {Slave\hspace{.5em}threads}
			\linebreak Depending on target Slave status..
			\begin{description}
				\item[DISCONNECTED: ] Wait for Slave to be marked by Listener thread
				\item[AVAILABLE: ] Attempt handshake to connect. Mark as disconnected 
					upon failure
				\item[CONNECTED: ] Listen for messages, count timeouts when applicable,
					fetch commands from user
			\end{description}
	\end{itemize}
	&
	\begin{itemize}[leftmargin=*]
		\item {Listener\hspace{.5em}thread:}
			\linebreak Listen for broadcasts. Depending on broadcast type...
			\begin{description}
				\item[Standard: ] If disconnected, send reply; if connected, reset Master
				timeout counter.
				\item[*Master timeout:] Ping Master before assuming disconnection
				\item[*Network timeout:] Ping self before rebooting
				\item[Update broadcast:] Shutdown Processor and reboot
				\item[Shutdown broadcast:] Shutdown Processor and reboot
				\item[Launch application:] (For Bootloader) Ignore when in MkII; launch
				MkII when in Bootloader
			\end{description}
		\item {MISO\hspace{.5em}thread:}
			\linebreak Send updates to Master when connected:
			\begin{itemize}
				\item Fetch updates from Processor, if any, or send empty message to
					maintain connection
				\item Send ping requests when flagged by Listener
				\item Remain idle when disconnected. NOTE: Empty processor Queue
			\end{itemize}
		\item {MOSI\hspace{.5em}thread:}
			\begin{itemize}
				\item Listen for messages from Master (when connected) and add them to
				Processor buffer
				\item Reset timeout counter whenever a Message is received
			\end{itemize}
	\end{itemize}
	\\
	\hline
	\multicolumn{2}{|c|}{\textbf{Connection and Disconnection}}\\
	\hline
	\begin{itemize}
		\item Use broadcast thread to keep Slave connected
		\item Use MISO-side of Slave thread to listen for periodic Slave-side updates
			to know when to assume disconnection
		\item NOTE: Send Disconnect message to Slave when assuming disconnection
		\item Send multiple MOSI messages (use index)
	\end{itemize}
	&
	\begin{itemize}
		\item Use listener thread to know if Master is still connected (based on
			broadcast)
		\item Ping Master when considering disconnection
		\item Send Disconnect message to Master when assuming disconnection
		\item Ping self to check network status before assuming network error and 
			rebooting
		\item Also reset Master timeout counter upon reception in MOSI thread
		\item Shutdown Processor when assuming disconnection from Master
	\end{itemize}
	\\
	\hline

\end{longtable}
\end{center}

\pagebreak

\begin{center}
\begin{longtable}{| p{.45\textwidth} | p{.45\textwidth} |}
	
	\hline
	\multicolumn{2}{|c|}{\textsc{Message formats}}\\
	\hline
	\textbf{MOSI} & \textbf{MISO}\\
	\hline
	\multicolumn{2}{|c|}{\textbf{Broadcast-side}}\\
	\hline
	\begin{itemize}
		\item {Standard\hspace{.5em}broadcast:} \
		\linebreak\texttt{\underline{N|}PASSCODE\underline{|}M\_L\_PORT}
		\item {Update\hspace{.5em}broadcast:} \
		\linebreak\texttt{\underline{U|}PASSCODE\underline{|}M\_L\_PORT\underline{|}FILE\_NAME\underline{|}FILE\_SIZE\_BYTES}
		\item {Shutdown\hspace{.5em}broadcast:} \
		\linebreak\texttt{\underline{R|}PASSCODE}
		\item {Launch\hspace{.5em}MkII:} \
		\linebreak\texttt{\underline{L|}PASSCODE}
		\item {Disconnect\hspace{.5em}MkII:} \
		\linebreak\texttt{\underline{X|}PASSCODE}
		\item {Delay\hspace{.5em}Bootloader:} \
		\linebreak\texttt{\underline{E|}PASSCODE}
	\end{itemize}
	&
	\begin{itemize}
		\item {Standard\hspace{.5em}broadcast\hspace{.5em}reply\hspace{.5em}(MkII):} \
		\linebreak\texttt{\underline{A|}PASSCODE\underline{|}S\_MAC\underline{|N|}S\_MISO\_P\underline{|}S\_MOSI\_P\underline{|}VERSION}
		\item {Error\hspace{.5em}(MkII Listener):} \
		\linebreak\texttt{\underline{A|}PASSCODE\underline{|}S\_MAC\underline{|E|}ERROR\_MESSAGE}
		\item {Error\hspace{.5em}(Bootloader):} \
		\linebreak\texttt{\underline{B|}PASSCODE\underline{|}S\_MAC\underline{|E|}ERROR\_MESSAGE}
		\item {Standard\hspace{.5em}broadcast\hspace{.5em}reply\hspace{.5em}(Bootloader):} \
		\linebreak\texttt{\underline{B|}PASSCODE\underline{|}S\_MAC\underline{|N}}
	\end{itemize}
	\\
	\hline
	\multicolumn{2}{|c|}{\textbf{Communications and Control}}\\*
	\hline
	\begin{itemize}
	{\scriptsize
		\item {Set\hspace{.5em}DC:}\
		\linebreak\texttt{MOSI\_INDEX\underline{|S|D:}DC\underline{:}000000000000000000000}
		\linebreak Here each character in the string of zeroes corresponds to a fan in the
		target Slave's array. A `1' means the fan is to be set to the specified DC, and 
		a `0' means it is to be left unchanged.
		\item {Chase\hspace{.5em}RPM:}\
		\linebreak\texttt{MOSI\_INDEX\underline{|S|C:}RPM\underline{:}000000000000000000000}
		\linebreak See ``Set DC" for the meaning of the string of zeroes.
		\item {Handshake:}\
		\linebreak\texttt{\underline{0|H|}COMMS\_CONFIG\underline{|}ARRAY\_CONFIG}
		\linebreak COMMS\_CONFIG is a comma-separated list with the following:
		\begin{enumerate}
			\item MISO port
			\item MOSI port
			\item Period (ms)
			\item Broadcast period (ms)
			\item Max. Master timeouts
		\end{enumerate}
		ARRAY\_CONFIG is a space-separated list with the following:
		\begin{enumerate}
			\item Fan mode
			\item Num. active fans
			\item PWM frequency (Hz)
			\item Counter counts
			\item Pulses per rotation 
			\item Max. RPM
			\item Min. RPM
			\item Min. DC
			\item Chaser tolerance (\%)
			\item Max. fan timeout
			\item PWM pinout
			\item Tach. pinout
			\end{enumerate}
		\item {Disconnect:}\
		\linebreak\texttt{MOSI\_INDEX\underline{|X}}
		\item {Reboot:}\
		\linebreak\texttt{MOSI\_INDEX\underline{|Z}}
		\item {Reset\hspace{.5em}index:}\
		\linebreak\texttt{MOSI\_INDEX\underline{|I}}
		\linebreak Slave will reset its MOSI index to 0.
		\item {Ping:}\
		\linebreak\texttt{MOSI\_INDEX\underline{|P}}
		\item {PSU:}\
		\linebreak\texttt{MOSI\_INDEX\underline{|S|}W:0}
		\linebreak Here the last character will be a 1 or 0 depending on the desired PSU			state (1 for ON and 0 for OFF). The PSU will be turned on upon startup and off
			upon shutdown and rebootby default.
		
	}
	\end{itemize}
	&
	\begin{itemize}
		\item {Maintain\hspace{.5em}connection:}\
		\linebreak\texttt{MISO\_INDEX\underline{|M}}
		\linebreak Sent to Master when there are no updates from Processor, but a MISO
		message is due to maintain connection.
		\item {Standard\hspace{.5em}update:}\
		\linebreak\texttt{MISO\_INDEX\underline{|T|}DATA\_INDEX\underline{|}RPMS\underline{|}DUTY\_CYCLES}
		\linebreak Here DUTY\_CYCLES and RPMS are comma-separated lists of the DC and
		RPM values of each fan in the array, in order. Negative values will be used for
		RPMS of fans being ``Chased."
		\item {Error\hspace{.5em}(MkII\hspace{.5em}MISO):}\
		\linebreak\texttt{MISO\_INDEX\underline{|E|}ERROR\_MESSAGE}
		\item {Handshakehspace{.5em}confirmation:}\
		\linebreak\texttt{MISO\_INDEX\underline{|H}}
		\linebreak For Slave-side exception handling and documenting.
		\item {Ping\hspace{.5em}request:}\
		\linebreak\texttt{MISO\_INDEX\underline{|P}}
		\item {MISO\hspace{.5em}index\hspace{.5em}reset:}\
		\linebreak\texttt{MISO\_INDEX\underline{|I}}
		\linebreak Master will reset its MISO index to 0.
		
	\end{itemize}
	\\
	\hline
	\pagebreak
	\hline
	\multicolumn{2}{|c|}{\textbf{Legend}}\\*
	\hline
	\begin{description}
		\item[N] ``NORMAL" i.e. Standard broadcast
		\item[U] ``UPDATE" i.e. Update broadcast
		\item[R] ``REBOOT" i.e. Reboot MCU
		\item[L] ``LAUNCH" i.e. Launch MkII
		\item[S] ``STANDARD" i.e. Standard command for Processor
		\item[D] ``DUTY CYCLE" i.e. Set Duty Cycle
		\item[C] ``CHASE" i.e. Chase RPM
		\item[H] ``HANDSHAKE" i.e. Handshake to start connection
		\item[X] ``DISCONNECT" i.e. Assume disconnection (Shutdown Processor)
		\item[Z] ``REBOOT" i.e. Reboot MCU
		\item[I] ``INDEX" i.e. Reset MISO Index
		\item[W] ``POWER" i.e. Power PSU
		\item[E] ``DELAY" i.e Reset bootloader timeout
	\end{description}
	&
	\begin{description}
		\item[A] ``APPLICATION" i.e. Message from MkII
		\item[B] ``BOOTLOADER" i.e. Message from Bootloader
		\item[M] ``MAINTAIN" i.e. Maintain connection
		\item[T] ``STANDARD" i.e. Standard update message
		\item[E] ``ERROR" i.e. Error message
		\item[P] ``PING" i.e. Ping request
		\item[I] ``INDEX" i.e. MISO index reset
	\end{description}
	\\
	\hline

\end{longtable}
\end{center}

\begin{center}
\renewcommand{\labelitemi}{-}
\begin{longtable}{| p{.92\textwidth} |}
\hline
\multicolumn{1}{|c|}{\textsc{To Do}}\\*
\hline
\multicolumn{1}{|l|}{Mon. 6/25/18 - Tue. 6/25/18}\\*
\hline
\begin{enumerate}
	\item \underline{Fix Bootloader 404 and empty file bugs}
	\item Add missing pinout, PSU pins and external LED pins
	\item Add placeholder for runtime pinout configuration
	\item \underline{Implement new message standard}
		\linebreak\indent Among other things...
		\begin{itemize}
			\item Receive S.Error and B.Error messages in both Slave threads
				and listener thread
			\item Use extra warnings in the event of a Bootloader error
		\end{itemize}
	\item \textbf{\underline{Implement Slave self-pinging}}
\end{enumerate}\\
\hline
\multicolumn{1}{|l|}{Wed. 6/25/18 - Fri 6/29/18}\\*
\hline
\begin{enumerate}
	\item Implement runtime pinout configuration
	\item Implement `efficient" tachometer
	\item Implement Master-side firmware uploads
	\item ``Fully" modularize Master
	\item Implement ``verifications" and shutdown button
	\item Implement user configuration and ``null" settings
	\item \textbf{Implement multiprocessing}
\end{enumerate}\\
\hline
\multicolumn{1}{|l|}{Mon. 7/2/18}\\*
\hline
\begin{enumerate}
	\item Fix PWM \texttt{read()} precision
	\item Fix PWM resolution
	\item Fix Chaser
	\item \textbf{Fix RPM spikes (if applicable)}
	\item Fix Processor thread-safety
\end{enumerate}\\
\hline
\multicolumn{1}{|l|}{Tue. 7/3/18 - Fri. 7/6/18}\\*
\hline
\begin{enumerate}
	\item + Fix S-side data types
	\item + Implement index resets (including \texttt{dataIndex}!)
	\item + Strong processor checks
	\item + PSU auto on/off setting
	\item + Input and output sockets in Master
	\item + Implement hotkeys
	\item + Implement plotter
	\item + Document
	\item + Add "help" section
	\item + Compile Master-side
	\item + Credits and licensing (And comments!)
\end{enumerate}\\

\hline
\end{longtable}
\end{center}

\end{document}
